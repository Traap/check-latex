\documentclass{article}%

  % Document preamble.
  \usepackage{catchfile}%
  \usepackage{dirtree}%
  \usepackage{forest}%

  \newcommand{\getEnv}[2][]{%
    \CatchFileEdef%
      {\temp}%
      {"|kpsewhich --var-value #2 2>/dev/null"}{\endlinechar=-1}%
    \if%
      \relax\detokenize{#1}\relax\temp%
    \else%
      \let#1\temp%
    \fi%
  }%

  \newcommand{\echoValues}[2]{\textbf{#1:}\newline[#2]}%

  \getEnv[\texDist]{TEXMFDIST}%
  \getEnv[\wsl]{WSL_DISTRO_NAME}%
  \getEnv[\autodocPath]{AUTODOCPATH}%

% The document
\begin{document}%

\section{Greetings}
Hello \LaTeX!

\section{Commands \& Environment}\label{sec:side-effects}%
Confirm \textbf{echoValues} \& \textbf{getEnv} commands work.

\begin{itemize}%
  \item \echoValues{TEXMFDIST}{\texDist}%
  \item \echoValues{WSL\_DISTRO\_NAME}{\wsl}%
  \item \echoValues{AUTODOCPATH}{\autodocPath}%
\end{itemize}

\section{dirtree package}%
\dirtree{%
.1 nvim.
.2 init.lua.
.2 LICENSE.
.2 lua.
.3 bootstrap.lua.
.3 plugins.lua.
.3 traap.
.4 chalk-base16.lua.
.4 compe-completion.lua.
.4 hightlights.lua.
.4 init.lua.
.4 keybindings.lua.
.4 language-servers.lua.
.4 nvim-tree.lua.
.4 nvim-web-devicons.lua.
.4 os-check.lua.
.4 settings.lua.
.4 telescope-nvim.lua.
.4 treesitter-nvim.lua.
.4 wiki-vim.lua.
.2 package-lock.json.
.2 README.md.
.3 en.utf-8.add.
.3 en.utf-8.add.spl.
}%

\section{forest package}%
\begin{forest} baseline, for tree=draw
  [Head Coach
    [Offensive Coordinator
      [Quarterbacks]
      [Offensive Line]
      [Running backs]
      [Receivers]
    ]
    [Defensive Coordinator
      [Special Teams]
      [Line backers]
      [Corners \& Safeties]
    ]
  ]
\end{forest}

\end{document}%
